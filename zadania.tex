\documentclass[11pt]{article}


\usepackage[left=2.00cm, right=2.00cm, top=2.00cm, bottom=2.00cm]{geometry}
\usepackage{polski}
\usepackage[utf8]{inputenc}
\usepackage{listings}
%opening
\title{
	\textbf{
		\LARGE{Algorytmy i struktury danych}\linebreak \\
		\large{Zadania obowiązkowe} \\
		\normalsize{Informatyka WIEiT - 2018/2019 }} \\
}
\author{}
\date{}
\renewcommand{\thesection}{\arabic{section}.}

\begin{document}
\lstset{language=C++,
	texcl=true,
	inputencoding=utf8,
	extendedchars=\true,
	}


\maketitle

\section{Ćwiczenia - Sortowania proste}
	Brak zadań obowiązkowych
\section{Ćwiczenia - MergeSort}

\begin{enumerate}
	\item Implementacja algorytmu MergeSort dla sortowania list
	
	\item Proszę zaproponować/zaimplementować algorytm scalający k posortowanych tablic o łącznej długości n
	w jedną posortowaną tablicę w czasie $O(n*log(k))$.
	
	\item Proszę zaproponować strukturę przechowującą liczby naturalne, w której operacje:
	Insert i GetMedian mają złożoność $O(log(n))$.
	Proszę zaimplementować w/w operacje.
	
	\item Proszę zaimplementować algorytm zliczający liczbę inwersji w tablicy
	(Inwersja to para indeksów $i,j$ taka, że $ i<j $ oraz $T[i]>T[j] $)
\end{enumerate}

\section{Ćwiczenia - QuickSort}

\begin{enumerate}
	\item Proszę zaimplementować algorytm QuickSort do sortowania listy jednokierunkowej.
	
	\item Proszę zaimplementować algorytm, który w czasie liniowym sortuje tablicę A zawierającą n liczb ze zbioru $ 0,1, ... ,n^2-1 $.
	
	\item Mamy serię pojemników z wodą, połączonych (każdy z każdym) rurami.
	Pojemniki maja kształty prostokątów (2d), rury nie maja objętości (powierzchni).
	Każdy pojemnik opisany jest przez współrzędne lewego górnego rogu i prawego dolnego rogu.
	Wiemy, ze do pojemników nalano A wody (oczywiście woda rurami spłynęła do najniższych pojemników).
	Obliczyć ile pojemników zostało w pełni zalanych.
	
	\item Dany jest ciąg przedziałów domkniętych $ [a_{1}, b_{1}], . . . ,[a_{n}, b_{n}]$.
	Proszę zaproponować algorytm, który znajduje taki przedział $[a_{t}, b_{t}]$,
	w którym w całości zawiera się jak najwięcej innych przedziałów.
\end{enumerate}

\newpage
\section{Ćwiczenia - Zastosowania sortowań}

\begin{enumerate}
	\item Dana jest posortowana tablica \texttt{int A[N]} oraz liczba $x$.
	Napisać program, który stwierdza czy istnieją indeksy $i$ oraz $j$,
	takie że $A[i]+A[j]=x$ (powinno działać w czasie $O(N)$).
	
	\item Zaimplementować algorytm, który dla tablicy \texttt{int A[N]} wyznacza
	rekurencyjną medianę median (magiczne piątki).
	
	\item Mamy daną tablicę A z n liczbami. Proszę zaproponować algorytm
	o złożoności $O(n)$, który stwierdza, czy w tablicy ponad połowa elementów
	ma jednakową wartość.
	
	\item Proszę zaproponować algorytm sortujący ciąg słów o różnych długościach
	w czasie proporcjonalnym do sumy długości tych słów.
\end{enumerate}

\section{Ćwiczenia - Struktury danych}

\begin{enumerate}
	\item Proszę zaimplementować dodawanie elementu do SkipListy.
	
	\item Proszę zaimplementować kolejkę przy użyciu dwóch stosów.
	
	
\end{enumerate}

\section{Ćwiczenia - Tablice z haszowaniem}

\begin{enumerate}
	
	\item Proszę zaimplementować następujące operacje na tablicy z haszowaniem:
	\begin{itemize}
		\item[$-$] wstawianie
		\item[$-$] usuwanie
		\item[$-$] wyszukiwanie
		\item[$-$] reorganizacja (usunięcie kluczy zaznaczonych do skasowania)
	\end{itemize}
	
	\item Dana jest nieposortowana tablica \texttt{int A[N]} oraz liczba $x$.
	Proszę napisać funkcję, która sprawdza na ile sposobów można przedstawić
	$x$ jako sumę $A[i]+A[j]$ takiego że $i<j$.
		
\end{enumerate}

\section{Ćwiczenia - Drzewa BST}

\begin{enumerate}
	
	\item Proszę podać modyfikację drzewa BST, która pozwala na efektywne
	wykonywanie następujących operacji:
	\begin{enumerate}
		\item znalezienie i-tego co do wielkości elementu w drzewie BST
		\item wyznaczenie, którym co do wielkości w drzewie jest zadany węzeł
	\end{enumerate}
	Proszę zaimplementować obie operacje.
	
\end{enumerate}
\section{Ćwiczenia - Drzewa czerwono-czarne}

\begin{enumerate}
	
	\item Zaimplementować funkcję, która koloruje węzły drzewa aby spełniało warunek drzewa RB,
	funkcja powinna zwracać informację czy udało się pokolorować drzewo.
	\item Udowodnić, że każde drzewo AVL jest drzewem RB.
	
\end{enumerate}
\section{Ćwiczenia - Drzewa B-tree}

\begin{enumerate}
	
	\item Dana jest tablica \texttt{bool A[N][N];} Gracz początkowo znajduje się na (zadanej) pozycji $(x, y)$,
	dla której zachodzi \texttt{A[y][x] == true}. Z danej pozycji wolno bezpośrednio przejść jedynie na pozycję,
	której dokładnie jedna współrzędna różni się o 1, oraz której wartość w tablicy $A$ wynosi \texttt{true}.
	Proszę napisać funkcję obliczający do ilu różnych pozycji może dojść gracz startując z zadanej pozycji $(x,~y)$.
	
	\item Dana jest struktura węzła drzewa B-tree przechowującego unikalne klucze: \newline
\begin{lstlisting}
	struct node {
		int n;			// liczba kluczy zawarta w węźle
		int key[N];		// tablica kluczy w węźle
		node* child[N+1];	// wskaźniki do synów węzła
		bool leaf;		// czy węzeł jest liściem
	};
\end{lstlisting}

	Proszę napisać funkcję \texttt{bool is\_b\_tree(node* p);} sprawdzającą czy
	wskaźnik $p$ wskazuje na poprawne drzewo B-tree.
	
	
\end{enumerate}
\section{Ćwiczenia - Grafy - BFS, DFS}
\begin{enumerate}
		
	\item Proszę zaimplementować następujące algorytmy:
	\begin{enumerate}
		\item Sprawdzanie czy graf jest dwudzielny
		\item Policzyć liczbę spójnych składowych w grafie
	\end{enumerate}
	Graf jest reprezentowany jako macierz sąsiedztwa albo listy sąsiadów albo listy krawędzi.
	
	\item Mówimy, że wierzchołek $t$ w grafie skierowanym jest ujściem, jeśli:	
	\begin{itemize}
		\item[$-$] z każdego innego wierzchołka $v$ istnieje krawędź z $v$ do $t$,
		\item[$-$] nie istnieje żadna krawędź wychodząca z $t$.
	\end{itemize}
	Podać algorytm znajdujący ujście (jeśli istnieje) przy macierzowej reprezentacji grafu.
	\end{enumerate}
\section{Ćwiczenia - Kolokwium nr 2}
	Brak zadań obowiązkowych
\section{Ćwiczenia - Grafy - Zastosowania algorytmów grafowych}

\begin{enumerate}
	
	\item Ścieżka Hamiltona przechodzi przez wszystkie wierzchołki w grafie, przez każdy dokładnie jeden raz. Proszę zaimplementować algorytm, który stwierdzi czy istnieje ścieżka Hamiltona w acyklicznym grafie skierowanym.
	\item Cykl Hamiltona o minimalnej sumie wag krawędzi nazywamy drogą komiwojażera. Dane są współrzędne N punktów na płaszczyźnie. Proszę zaimplementować funkcję	obliczającą dolne ograniczenie drogi komiwojażera.
	
		
{\small \textit{	Wskazówka: Jaki związek ma długość drogi komiwojażera z długością minimalnego drzewa rozpinającego?}}
\end{enumerate}

\section{Ćwiczenia - Grafy - Znajdywanie najkrótszej ścieżki w grafie}

\begin{enumerate}
	
	\item Zaimplementuj algorytm Dijkstry dla grafu w dowolnej postaci.
	\item Zmodyfikuj algorytm Dijkstry, aby móc odtworzyć najkrótszą ścieżkę między dwoma wierzchołkami grafu.
	
\end{enumerate}
	
\end{document}


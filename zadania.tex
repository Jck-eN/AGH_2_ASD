\documentclass[11pt]{article}


\usepackage[left=2.00cm, right=2.00cm, top=2.00cm, bottom=2.00cm]{geometry}
\usepackage{polski}
\usepackage[utf8]{inputenc} 
%opening
\title{
	\textbf{
		\LARGE{Algorytmy i struktury danych}\linebreak \\
		\large{Zadania obowiązkowe} \\
		\normalsize{Informatyka WIEiT - 2018/2019 }} \\
}
\author{}
\date{}
\renewcommand{\thesection}{\arabic{section}.}

\begin{document}

\maketitle

\section{Ćwiczenia - Sortowania proste}
	Brak zadań obowiązkowych
\section{Ćwiczenia - MergeSort}

\begin{enumerate}
	\item Implementacja algorytmu MergeSort dla sortowania list
	
	\item Proszę zaproponować/zaimplementować algorytm scalający k posortowanych tablic o łącznej długości n
	w jedną posortowaną tablicę w czasie $O(n*log(k))$.
	
	\item Proszę zaproponować strukturę przechowującą liczby naturalne, w której operacje:
	Insert i GetMedian mają złożoność $O(log(n))$.
	Proszę zaimplementować w/w operacje.
	
	\item Proszę zaimplementować algorytm zliczający liczbę inwersji w tablicy
	(Inwersja to para indeksów $i,j$ taka, że $ i<j $ oraz $T[i]>T[j] $)
\end{enumerate}

\section{Ćwiczenia - QuickSort}

\begin{enumerate}
	\item Proszę zaimplementować algorytm QuickSort do sortowania listy jednokierunkowej.
	
	\item Proszę zaimplementować algorytm, który w czasie liniowym sortuje tablicę A zawierającą n liczb ze zbioru $ 0,1, ... ,n^2-1 $.
	
	\item Mamy serię pojemników z wodą, połączonych (każdy z każdym) rurami.
	Pojemniki maja kształty prostokątów (2d), rury nie maja objętości (powierzchni).
	Każdy pojemnik opisany jest przez współrzędne lewego górnego rogu i prawego dolnego rogu.
	Wiemy, ze do pojemników nalano A wody (oczywiście woda rurami spłynęła do najniższych pojemników).
	Obliczyć ile pojemników zostało w pełni zalanych.
	
	\item Dany jest ciąg przedziałów domkniętych $ [a_{1}, b_{1}], . . . ,[a_{n}, b_{n}]$.
	Proszę zaproponować algorytm, który znajduje taki przedział $[a_{t}, b_{t}]$,
	w którym w całości zawiera się jak najwięcej innych przedziałów.
\end{enumerate}

\newpage
\section{Ćwiczenia - Zastosowania sortowań}

\begin{enumerate}
	\item Dana jest posortowana tablica \texttt{int A[N]} oraz liczba $x$.
	Napisać program, który stwierdza czy istnieją indeksy $i$ oraz $j$,
	takie że $A[i]+A[j]=x$ (powinno działać w czasie $O(N)$).
	
	\item Zaimplementować algorytm, który dla tablicy \texttt{int A[N]} wyznacza
	rekurencyjną medianę median (magiczne piątki).
	
	\item Mamy daną tablicę \texttt{A} z n liczbami. Proszę zaproponować algorytm
	o złożoności $O(n)$, który stwierdza, czy w tablicy ponad połowa elementów
	ma jednakową wartość.
	
	\item Proszę zaproponować algorytm sortujący ciąg słów o różnych długościach
	w czasie proporcjonalnym do sumy długości tych słów.
\end{enumerate}

\section{Ćwiczenia - Struktury danych}

\begin{enumerate}
	\item Proszę zaimplementować dodawanie elementu do SkipListy.
	
	\item Proszę zaimplementować kolejkę przy użyciu dwóch stosów.
	
	
\end{enumerate}

\section{Ćwiczenia - Tablice z haszowaniem}

\begin{enumerate}
	
	\item Proszę zaimplementować następujące operacje na tablicy z haszowaniem:
	\begin{itemize}
		\item[$-$] wstawianie
		\item[$-$] usuwanie
		\item[$-$] wyszukiwanie
		\item[$-$] reorganizacja (usunięcie kluczy zaznaczonych do skasowania)
	\end{itemize}
	
	\item Dana jest nieposortowana tablica \texttt{int A[N]} oraz liczba $x$.
	Proszę napisać funkcję, która sprawdza na ile sposobów można przedstawić
	$x$ jako sumę $A[i]+A[j]$ takiego że $i<j$.
		
\end{enumerate}

\section{Ćwiczenia - Drzewa BST}

\begin{enumerate}
	
	\item Proszę podać modyfikację drzewa BST, która pozwala na efektywne
	wykonywanie następujących operacji:
	\begin{enumerate}
		\item znalezienie i-tego co do wielkości elementu w drzewie BST
		\item wyznaczenie, którym co do wielkości w drzewie jest zadany węzeł
	\end{enumerate}
	Proszę zaimplementować obie operacje.
	
\end{enumerate}


\end{document}
